% To write matrix symbols
\usepackage{dsfont}

% Have a bold symbol for the vectors
\usepackage{amsmath}

% --- Cross over a math symbol
%\usepackage[makeroom]{cancel}

% --- Option to have nice matrices
\usepackage{array}

% Greek math symbols in bold
%\usepackage{unicode-math}


% Nice arrows in mathmode
\usepackage{marvosym} % \MVRIGHTarrow

% Long right arrow with text
\usepackage{extarrows}


% --- Use SI-units for parameters
% \usepackage{SIunits}
% \usepackage{sistyle}
%\usepackage{siunitx}


%%Math shortcuts

%\begin{comment}
% Norms
\DeclarePairedDelimiter\abs{\lvert}{\rvert}%
\DeclarePairedDelimiter\norm{\lvert}{\rvert}%
\newcommand{\norminf}[1]{\norm{#1}_\infty}
\newcommand{\dist}{\operatorname{dist}}

% Common sets
\newcommand{\R}{\mathbb{R}} % Real
\newcommand{\N}{\mathbb{N}} % Natural
\newcommand{\Z}{\mathbb{Z}} % Integer
\newcommand{\V}{\mathbb{V}} % Voronoi partition
\newcommand{\W}{\mathcal{W}} % Workspace
%\newcommand{\B}{\mathbb{B}} % Ball
\newcommand{\X}{\mathcal{X}} % MAS configuration

% Relation Symbols
\newcommand{\defeq}{\mathop{\vcentcolon=}}
\newcommand{\eqdef}{\mathop{=\vcentcolon}}
\newcommand{\meq}{\mathop{\stackrel{!}{=}}}

% Subsets, subspaces and some linear algebra
\newcommand{\interior}{\operatorname{int}}
\newcommand{\depth}{\operatorname{depth}}
\newcommand{\linspan}{\operatorname{span}}
\newcommand{\conv}{\operatorname{conv}}
\newcommand{\aff}{\operatorname{aff}}
\newcommand{\affdim}{\operatorname{affdim}}
\newcommand{\relint}{\operatorname{relint}}
\newcommand{\CS}{\operatorname{CS}}
\newcommand{\image}{\operatorname{image}}
\newcommand{\row}{\operatorname{row}}

% Special symbols
\newcommand{\xig}{\bar{x}_i^{\text{g}}}

% Optimization
\newcommand{\opt}[1]{ #1^\star}
\newcommand{\argmin}{\arg \min}

% Matrices
\newcommand{\bmat}[1]{\begin{bmatrix}#1\end{bmatrix}}
\newcommand{\mmat}[1]{\begin{matrix}#1\end{matrix}}
\newcommand{\diag}{\mathrm{diag}}
\newcommand{\rank}{\mathrm{rank}}

% Graph commands
\newcommand{\Vertices}{\mathcal{V}}
\newcommand{\Graph}{\mathcal{G}}
\newcommand{\Edges}{\mathcal{E}}
\newcommand{\E}{\mathcal{E}}


% Common vectors
\newcommand{\0}{\mymathbb{0}}
\newcommand{\1}{\mymathbb{1}}

% Some operators
\newcommand{\average}{\operatorname{average}}

%\end{comment}

%Allow for wider blocks
\newenvironment<>{varblock}[2][1.1\textwidth]
{%
  \setlength{\textwidth}{#1}
  \begin{actionenv}#3%
    \def\insertblocktitle{#2}%
    \par%
    \usebeamertemplate{block begin}
  {\par}
  \usebeamertemplate{block end}%
  \end{actionenv}
}



% -------------------
% --- Commands for the indices marking

\newcommand*{\tinysup}[1]{\ensuremath{^{\tiny #1}}}
\newcommand*{\tinysub}[1]{\ensuremath{_{\tiny #1}}}
\newcommand*{\supt}[1]{\ensuremath{^{\tiny #1}}}
\newcommand*{\subt}[1]{\ensuremath{_{\tiny #1}}}


% Derivative in point notation
\newcommand{\at}[2][]{#1\big|_{#2}}
% \newcommand\at[2]{\left.#1\right|_{#2}}



% Defining the normal-symbol
\newcommand{\normal}[1]{\lvert #1 \rvert}

%Parallel sign:
\newcommand{\parallelsum}{\mathbin{\|}}

%Question mark over equal sign:
\newcommand{\?}{\stackrel{\color{red}?\color{black}}{=}}


% --- Spherical harmonics and the potential functions

\newcommand{\shm}[2]{Y \subt{#1}\supt{#2}\left(\theta,\phi\right)}
 \newcommand{\shmN}[3]{Y \subt{#1}\supt{#2}\left(\theta_{#3},\phi_{#3}\right) }
  \newcommand{\shmc}[2]{\left[ Y \subt{#1'}\supt{#2'}\!\left(\theta,\phi\right) \right]^{\ast}}

 \newcommand{\christoffel}[3]{\ensuremath{\Gamma\supt{#1#2}\subt{#3}}}


\newcommand{\sphm}[2]{Y \subt{#1}\supt{#2}\left(\arccos\eta,\phi\right)}


% New definition of the electrostatic potential
\newcommand{\elPot}{\ensuremath{\psi\!\left(\mathbf{r}|\omega\right)}}
\newcommand{\elPotN}[1]{\ensuremath{\psi_{#1}\!\left(\mathbf{r}|\omega\right)}}

\newcommand{\elInc}{\ensuremath{\psi_{inc}\!\left(\mathbf{r}|\omega\right)}}

\newcommand{\elIncN}[1]{\ensuremath{\psi_{inc}^{(#1)}\!\left(\mathbf{r}|\omega\right)}}


\newcommand{\elTr}{\ensuremath{\psi_{tr}\!\left(\mathbf{r}|\omega\right)}}

\newcommand{\elTrN}[1]{\ensuremath{\psi_{tr}^{(#1)}\!\left(\mathbf{r}|\omega\right)}}

\newcommand{\elCons}[1]{\ensuremath{\psi_{0}^{(#1)}(\omega)}}

%Command for writing SH-integral
% \newcommand{\shInt}[2]{\ensuremath{\int_{#1}^{#2}\text{d}(\cos\theta)\int_0^{2\pi}\text{d}\phi}}

\newcommand{\shInt}[2]{\ensuremath{\int_{#1}^{#2}\text{d}\Omega}}



%--- Command for the lm'th multipole coefficient
\newcommand{\A}{\ensuremath{A_{\ell m}(\omega)}}

\newcommand{\An}[1]{\ensuremath{A_{\ell m}^{(#1)}(\omega)}}
\newcommand{\AnBar}[1]{\ensuremath{\bar{A}_{\ell m}^{(#1)}(\omega)}}

\newcommand{\B}{\ensuremath{B_{\ell m}(\omega)}}
\newcommand{\Bn}[1]{\ensuremath{B_{\ell m}^{(#1)}(\omega)}}
\newcommand{\BnBar}[1]{\ensuremath{\bar{B}_{\ell m}^{(#1)}(\omega)}}

\newcommand{\Bprime}{\ensuremath{B_{\ell^\prime m^\prime}(\omega)}}




%e:
\newcommand{\e}{\mathrm{e}}

%Make the \vec command to bold-math symbol
\renewcommand{\vec}[1]{\boldsymbol{\mathbf{#1}}}
% --- Define \dvec and \ddvec for dotted and double-dotted vectors.



% The infinitessimal symbol

\newcommand{\dx}[1][]{%
 %  \ifthenelse{ \equal{#1}{} }
  %    {\ensuremath{\;\mathrm{d}x}}
      {\ensuremath{\;\mathrm{d}#1}}
}


% Tensor symbol
\DeclareFontFamily{OMS}{oasy}{\skewchar\font48 }
\DeclareFontShape{OMS}{oasy}{m}{n}{%
         <-5.5> oasy5     <5.5-6.5> oasy6
      <6.5-7.5> oasy7     <7.5-8.5> oasy8
      <8.5-9.5> oasy9     <9.5->  oasy10
      }{}
\DeclareFontShape{OMS}{oasy}{b}{n}{%
       <-6> oabsy5
      <6-8> oabsy7
      <8->  oabsy10
      }{}
\DeclareSymbolFont{oasy}{OMS}{oasy}{m}{n}
\SetSymbolFont{oasy}{bold}{OMS}{oasy}{b}{n}

\DeclareMathSymbol{\smallleftarrow}     {\mathrel}{oasy}{"20}
\DeclareMathSymbol{\smallrightarrow}    {\mathrel}{oasy}{"21}
\DeclareMathSymbol{\smallleftrightarrow}{\mathrel}{oasy}{"24}


\newcommand{\tensor}[1]{\overset{\scriptscriptstyle\smallleftrightarrow}{#1}}


\newcommand{\vectens}[1]{\overset{\scriptscriptstyle\smallrightarrow}{#1}}

%Set color around equations
% \newcommand{\boxedeq}[2]{\begin{empheq}[box={\fboxsep=6pt\fbox}]{align}\label{#1}#2\end{empheq}}
% \newcommand{\coloredeq}[2]{\begin{empheq}[box=\colorbox{my_raspberry}]{align}\label{#1}#2\end{empheq}}
