\usepackage[utf8]{inputenc}
\usepackage{amsmath}
\usepackage{amsthm}

% to comment out code
\usepackage{comment}

%% Colors
%% ------

\definecolor{sg_blue}{RGB}{0, 149, 212}
\definecolor{sg_darkblue}{RGB}{23, 66, 140}
\definecolor{sg_green}{RGB}{79, 176, 181}
\definecolor{sg_orange}{RGB}{237, 102, 8}
\definecolor{sg_red}{RGB}{227, 0, 59}

\definecolor{my_purple}{RGB}{158, 93, 182}
\definecolor{my_nightblue}{RGB}{0, 55, 123}
\definecolor{my_blue}{RGB}{80, 130, 220}
\definecolor{my_duck}{RGB}{0, 125, 172}
\definecolor{my_turquoise}{RGB}{0, 140, 125}
\definecolor{my_forest}{RGB}{20, 150, 50}
\definecolor{my_apple}{RGB}{90, 180, 50}
\definecolor{my_yellow}{RGB}{249, 190, 36}
\definecolor{my_orange}{RGB}{255, 120, 15}
\definecolor{my_red}{RGB}{250, 50, 10}
\definecolor{my_raspberry}{RGB}{248, 15, 80}
\definecolor{my_brown}{RGB}{150, 95, 50}

\definecolor{my_lightgray}{RGB}{160, 160, 160} % before 200 200 200
\definecolor{my_black}{RGB}{10, 10, 10}
\definecolor{my_gray}{RGB}{100, 100, 100}
\definecolor{my_darkgray}{RGB}{50, 50, 50}
\definecolor{my_white_back}{RGB}{247, 247, 247}
\definecolor{my_white_edge}{RGB}{253, 253, 253}

% Colors meaning
% --------------

\definecolor{water}{named}{my_blue}
\colorlet{ethanol}{my_blue!50!my_raspberry}
\colorlet{propylene}{my_blue!50!my_forest}
\definecolor{dowfax}{named}{my_raspberry}
\definecolor{sds}{named}{my_nightblue}
%\definecolor{my_main}{named}{my_lightgray}
\definecolor{my_main}{named}{my_nightblue}
%\definecolor{my_main}{named}{sg_blue}
\colorlet{mixture}{water!50!propylene}

%% Beamer modifications
%% --------------------

% Use of the colors by beamer
% ---------------------------

\setbeamercolor{normal text}{fg=my_darkgray}
% To have a different back color for the picture
%\setbeamercolor{normal text}{fg=my_darkgray, bg=my_white_back}
\setbeamercolor{frametitle}{fg=my_nightblue}
\setbeamercolor{item}{fg=my_nightblue}
\setbeamercolor{framesubtitle}{fg=my_darkgray}
\setbeamercolor{pagenumber}{fg=my_darkgray}

% \setbeamercolor{normal text}{fg=my_darkgray, bg=my_white_back}
% \setbeamercolor{frametitle}{fg=my_darkgray}
% \setbeamercolor{item}{fg=my_darkgray}
% \setbeamercolor{framesubtitle}{fg=my_darkgray}
% \setbeamercolor{pagenumber}{fg=my_darkgray}


\setbeamercolor{alerted text}{fg=my_main}
\setbeamercolor{example text}{fg=my_raspberry}
\setbeamercolor{structure}{fg=my_main}
\setbeamercolor{background canvas}{parent=normal text}
%\setbeamercolor{background canvas}{parent=white}
\setbeamercolor{background}{parent=white}
%\setbeamercolor{background}{parent=background canvas}

\usefonttheme{default}
% font themes: default, professionalfonts, serif,
% structurebold, structureitalicserif, structuresmallcapsserif

% Add frame number in the navigation symbol
% -------------------
\setbeamertemplate{navigation symbols}{}
%\addtobeamertemplate{navigation symbols}{}{%
%	\usebeamerfont{footline}%
%	\usebeamercolor[fg]{footline}%
%	\hspace{1em}%
%	\insertframenumber/\inserttotalframenumber
%}

% New colors / from Franck
% ----------

%\newcommand{\greyemph}[1]{\textcolor{gray}{#1}}
%\newcommand{\dgreyemph}[1]{\textcolor{darkgray}{#1}}
%\newcommand{\citeemph}[1]{{\footnotesize \dgreyemph{[#1]}}}
%\newcommand{\defemph}[1]{{\usebeamercolor[fg]{frametitle} #1}}
%\newcommand{\remph}[1]{{\usebeamercolor[fg]{alerted text} #1}}

% Numbering and items
% -------------------

%\setbeamertemplate{footline}[frame number]
%\setbeamertemplate{itemize item}[circle]
%\setbeamertemplate{itemize subitem}[circle]




\usepackage[parfill]{parskip}
\usepackage[section]{placeins}
\usepackage{graphicx}




%To scale the footnote size properly
\usepackage{mathptmx}
\usepackage{anyfontsize}
\usepackage{t1enc}

%% Math imports
\usepackage{mathtools}
\usepackage{amsmath}
\usepackage{amsfonts}
\usepackage{amssymb}
\usepackage{soul}

\usepackage{setspace}
\usepackage{color}
\usepackage{listings}

%Add latex_code to the document





%\usepackage[marginparwidth=3cm]{geometry}
%\usepackage{marginnote}
\DeclareMathAlphabet{\mymathbb}{U}{bbold}{m}{n}

%To set the usetheme and the colors
%\usetheme{Madrid}
%\usetheme{default}

%Set the colors
%\usecolortheme{rose}
% color themes: albatross, beaver, beetle, crane, default, dolphin, dov, fly, lily, orchid, rose, seagull, seahorse, sidebartab, structure, whale, wolverine

%Set the text style
%\usefonttheme{serif}
% font themes: default, professionalfonts, serif, structurebold, structureitalicserif, structuresmallcapsserif

%If you want to change the text colors again
% \setbeamercolor{normal text}{fg=black,bg=white}
% \setbeamercolor{pagenumber}{fg=pagenumbercolor}
% \setbeamercolor{alerted text}{fg=red}
% \setbeamercolor{example text}{fg=blue!50!white}
% \setbeamercolor{structure}{fg=blue!70!white}
% \setbeamercolor{background canvas}{parent=normal text}
% \setbeamercolor{background}{parent=background canvas}
% \setbeamercolor{framesource}{fg=black}
% \setbeamerfont{framesource}{size=\small}

%% To write matrix symbols
\usepackage{dsfont}

% Have a bold symbol for the vectors
\usepackage{amsmath}

% --- Cross over a math symbol
%\usepackage[makeroom]{cancel}

% --- Option to have nice matrices
\usepackage{array}

% Greek math symbols in bold
%\usepackage{unicode-math}


% Nice arrows in mathmode
\usepackage{marvosym} % \MVRIGHTarrow

% Long right arrow with text
\usepackage{extarrows}


% --- Use SI-units for parameters
% \usepackage{SIunits}
% \usepackage{sistyle}
%\usepackage{siunitx}


%%Math shortcuts

%\begin{comment}
% Norms
\DeclarePairedDelimiter\abs{\lvert}{\rvert}%
\DeclarePairedDelimiter\norm{\lvert}{\rvert}%
\newcommand{\norminf}[1]{\norm{#1}_\infty}
\newcommand{\dist}{\operatorname{dist}}

% Common sets
\newcommand{\R}{\mathbb{R}} % Real
\newcommand{\N}{\mathbb{N}} % Natural
\newcommand{\Z}{\mathbb{Z}} % Integer
\newcommand{\V}{\mathbb{V}} % Voronoi partition
\newcommand{\W}{\mathcal{W}} % Workspace
%\newcommand{\B}{\mathbb{B}} % Ball
\newcommand{\X}{\mathcal{X}} % MAS configuration

% Relation Symbols
\newcommand{\defeq}{\mathop{\vcentcolon=}}
\newcommand{\eqdef}{\mathop{=\vcentcolon}}
\newcommand{\meq}{\mathop{\stackrel{!}{=}}}

% Subsets, subspaces and some linear algebra
\newcommand{\interior}{\operatorname{int}}
\newcommand{\depth}{\operatorname{depth}}
\newcommand{\linspan}{\operatorname{span}}
\newcommand{\conv}{\operatorname{conv}}
\newcommand{\aff}{\operatorname{aff}}
\newcommand{\affdim}{\operatorname{affdim}}
\newcommand{\relint}{\operatorname{relint}}
\newcommand{\CS}{\operatorname{CS}}
\newcommand{\image}{\operatorname{image}}
\newcommand{\row}{\operatorname{row}}

% Special symbols
\newcommand{\xig}{\bar{x}_i^{\text{g}}}

% Optimization
\newcommand{\opt}[1]{ #1^\star}
\newcommand{\argmin}{\arg \min}

% Matrices
\newcommand{\bmat}[1]{\begin{bmatrix}#1\end{bmatrix}}
\newcommand{\mmat}[1]{\begin{matrix}#1\end{matrix}}
\newcommand{\diag}{\mathrm{diag}}
\newcommand{\rank}{\mathrm{rank}}

% Graph commands
\newcommand{\Vertices}{\mathcal{V}}
\newcommand{\Graph}{\mathcal{G}}
\newcommand{\Edges}{\mathcal{E}}
\newcommand{\E}{\mathcal{E}}


% Common vectors
\newcommand{\0}{\mymathbb{0}}
\newcommand{\1}{\mymathbb{1}}

% Some operators
\newcommand{\average}{\operatorname{average}}

%\end{comment}

%Allow for wider blocks
\newenvironment<>{varblock}[2][1.1\textwidth]
{%
  \setlength{\textwidth}{#1}
  \begin{actionenv}#3%
    \def\insertblocktitle{#2}%
    \par%
    \usebeamertemplate{block begin}
  {\par}
  \usebeamertemplate{block end}%
  \end{actionenv}
}



% -------------------
% --- Commands for the indices marking

\newcommand*{\tinysup}[1]{\ensuremath{^{\tiny #1}}}
\newcommand*{\tinysub}[1]{\ensuremath{_{\tiny #1}}}
\newcommand*{\supt}[1]{\ensuremath{^{\tiny #1}}}
\newcommand*{\subt}[1]{\ensuremath{_{\tiny #1}}}


% Derivative in point notation
\newcommand{\at}[2][]{#1\big|_{#2}}
% \newcommand\at[2]{\left.#1\right|_{#2}}



% Defining the normal-symbol
\newcommand{\normal}[1]{\lvert #1 \rvert}

%Parallel sign:
\newcommand{\parallelsum}{\mathbin{\|}}

%Question mark over equal sign:
\newcommand{\?}{\stackrel{\color{red}?\color{black}}{=}}


% --- Spherical harmonics and the potential functions

\newcommand{\shm}[2]{Y \subt{#1}\supt{#2}\left(\theta,\phi\right)}
 \newcommand{\shmN}[3]{Y \subt{#1}\supt{#2}\left(\theta_{#3},\phi_{#3}\right) }
  \newcommand{\shmc}[2]{\left[ Y \subt{#1'}\supt{#2'}\!\left(\theta,\phi\right) \right]^{\ast}}

 \newcommand{\christoffel}[3]{\ensuremath{\Gamma\supt{#1#2}\subt{#3}}}


\newcommand{\sphm}[2]{Y \subt{#1}\supt{#2}\left(\arccos\eta,\phi\right)}


% New definition of the electrostatic potential
\newcommand{\elPot}{\ensuremath{\psi\!\left(\mathbf{r}|\omega\right)}}
\newcommand{\elPotN}[1]{\ensuremath{\psi_{#1}\!\left(\mathbf{r}|\omega\right)}}

\newcommand{\elInc}{\ensuremath{\psi_{inc}\!\left(\mathbf{r}|\omega\right)}}

\newcommand{\elIncN}[1]{\ensuremath{\psi_{inc}^{(#1)}\!\left(\mathbf{r}|\omega\right)}}


\newcommand{\elTr}{\ensuremath{\psi_{tr}\!\left(\mathbf{r}|\omega\right)}}

\newcommand{\elTrN}[1]{\ensuremath{\psi_{tr}^{(#1)}\!\left(\mathbf{r}|\omega\right)}}

\newcommand{\elCons}[1]{\ensuremath{\psi_{0}^{(#1)}(\omega)}}

%Command for writing SH-integral
% \newcommand{\shInt}[2]{\ensuremath{\int_{#1}^{#2}\text{d}(\cos\theta)\int_0^{2\pi}\text{d}\phi}}

\newcommand{\shInt}[2]{\ensuremath{\int_{#1}^{#2}\text{d}\Omega}}



%--- Command for the lm'th multipole coefficient
\newcommand{\A}{\ensuremath{A_{\ell m}(\omega)}}

\newcommand{\An}[1]{\ensuremath{A_{\ell m}^{(#1)}(\omega)}}
\newcommand{\AnBar}[1]{\ensuremath{\bar{A}_{\ell m}^{(#1)}(\omega)}}

\newcommand{\B}{\ensuremath{B_{\ell m}(\omega)}}
\newcommand{\Bn}[1]{\ensuremath{B_{\ell m}^{(#1)}(\omega)}}
\newcommand{\BnBar}[1]{\ensuremath{\bar{B}_{\ell m}^{(#1)}(\omega)}}

\newcommand{\Bprime}{\ensuremath{B_{\ell^\prime m^\prime}(\omega)}}




%e:
\newcommand{\e}{\mathrm{e}}

%Make the \vec command to bold-math symbol
\renewcommand{\vec}[1]{\boldsymbol{\mathbf{#1}}}
% --- Define \dvec and \ddvec for dotted and double-dotted vectors.



% The infinitessimal symbol

\newcommand{\dx}[1][]{%
 %  \ifthenelse{ \equal{#1}{} }
  %    {\ensuremath{\;\mathrm{d}x}}
      {\ensuremath{\;\mathrm{d}#1}}
}


% Tensor symbol
\DeclareFontFamily{OMS}{oasy}{\skewchar\font48 }
\DeclareFontShape{OMS}{oasy}{m}{n}{%
         <-5.5> oasy5     <5.5-6.5> oasy6
      <6.5-7.5> oasy7     <7.5-8.5> oasy8
      <8.5-9.5> oasy9     <9.5->  oasy10
      }{}
\DeclareFontShape{OMS}{oasy}{b}{n}{%
       <-6> oabsy5
      <6-8> oabsy7
      <8->  oabsy10
      }{}
\DeclareSymbolFont{oasy}{OMS}{oasy}{m}{n}
\SetSymbolFont{oasy}{bold}{OMS}{oasy}{b}{n}

\DeclareMathSymbol{\smallleftarrow}     {\mathrel}{oasy}{"20}
\DeclareMathSymbol{\smallrightarrow}    {\mathrel}{oasy}{"21}
\DeclareMathSymbol{\smallleftrightarrow}{\mathrel}{oasy}{"24}


\newcommand{\tensor}[1]{\overset{\scriptscriptstyle\smallleftrightarrow}{#1}}


\newcommand{\vectens}[1]{\overset{\scriptscriptstyle\smallrightarrow}{#1}}

%Set color around equations
% \newcommand{\boxedeq}[2]{\begin{empheq}[box={\fboxsep=6pt\fbox}]{align}\label{#1}#2\end{empheq}}
% \newcommand{\coloredeq}[2]{\begin{empheq}[box=\colorbox{my_raspberry}]{align}\label{#1}#2\end{empheq}}


%Add pagenumering
% \beamertemplatenavigationsymbolsempty
% \addtobeamertemplate{navigation symbols}{}{ \hspace{1em}    \usebeamerfont{footline}%
%   \insertframenumber / \inserttotalframenumber }

%% Bibliography
%\usepackage[backend=biber,style=ieee]{biblatex}
%\bibliography{manual}

% % To show references on the same slide:
% \usepackage[backend=bibtex]{biblatex}

% \makeatletter
% \def\blx@maxline{77}
% \makeatother

% % Resset footnote to be showed on each slide
%  \let\oldfootnote\footnote
% \renewcommand\footnote[1][]{\oldfootnote[frame,#1]}
% \makeatother

% % Use square-parenthesis as mark
% \makeatletter
% \renewcommand\@makefnmark{\hbox{\usebeamercolor[fg]{footnote mark}\usebeamerfont*{footnote mark}[\@thefnmark]}}
% \renewcommand\@makefntext[1]{\fontsize{4}{5}\selectfont {\usebeamercolor[fg]{footnote
%       mark}\usebeamerfont*{footnote
%       mark}[\@thefnmark]}\enspace\usebeamerfont*{footnote} #1}
% \makeatother



% To show references on the same slide:
\usepackage[backend=bibtex]{biblatex}

\makeatletter
\def\blx@maxline{77}
\makeatother

% Resset footnote to be showed on each slide
 \let\oldfootnote\footnote
\renewcommand\footnote[1][]{\oldfootnote[frame,#1]}
\makeatother

% Use square-parenthesis as mark
\makeatletter
\renewcommand\@makefnmark{\hbox{\usebeamercolor[fg]{footnote mark}\usebeamerfont*{footnote mark}[\@thefnmark]}}
\renewcommand\@makefntext[1]{\fontsize{4}{5}\selectfont {\usebeamercolor[fg]{footnote
      mark}\usebeamerfont*{footnote
      mark}[\@thefnmark]}\enspace\usebeamerfont*{footnote} #1}
\makeatother



% \let\svthefootnote\thefootnote
% \textheight 1in
% \newcommand\blankfootnote[1]{%
%   \let\thefootnote\relax\footnotetext{#1}%
%   \let\thefootnote\svthefootnote%
% }





% % Use square-parenthesis as mark
% \makeatletter
% \renewcommand\@makefnmark{\hbox{\@textsuperscript{\usebeamercolor[fg]{footnote mark}\usebeamerfont*{footnote mark}[\@thefnmark]}}}
% \renewcommand\@makefntext[1]{\@textsuperscript{\usebeamercolor[fg]{footnote
%       mark}\usebeamerfont*{footnote
%       mark}[\@thefnmark]}\enspace\usebeamerfont*{footnote} #1}
% \makeatother



% --- Add predefined colors
\usepackage{xcolor}
%\usepackage{listings}
\lstset
{
    language=[LaTeX]TeX,
    breaklines=true,
    basicstyle=\tt\scriptsize,
    keywordstyle=\color{my_black},
    identifierstyle=\color{my_nightblue},
    basicstyle=\footnotesize %\tiny, %or \small or \footnotesize etc.
}

%Add tikz functionlaity to the document
% \usepackage{tikz}
% \usepackage{showexpl}
% \lstset{
%     language=[LaTeX]TeX,
%     breaklines=true,
%     basicstyle=\tt\scriptsize,
%     keywordstyle=\color{blue},
%     identifierstyle=\color{magenta},
% }

%Set nice figures
\usepackage{graphicx}
%\usepackage{subcaption}

\usepackage{calc} % Allow use of mathematical expressions to set lenght units etc.
\usepackage{url} % Break long urls over several lines


\usepackage{pgfplots}
\pgfplotsset{compat=newest}

%% Tikz
\usepackage{tikz,overpic}
\usetikzlibrary{fit,shapes.misc}
\usetikzlibrary{plotmarks}


%set the arrow style
\tikzset{>=latex}
\usetikzlibrary{arrows.meta,positioning,decorations.markings,backgrounds}
%\usetikzlibrary{arrows.meta}
\usepgfplotslibrary{patchplots}
\usepackage{grffile}
\pgfplotsset{plot coordinates/math parser=false}
\setbeamercovered{dynamic}

% Figure width parameter
\newlength\fwidth
\newlength\fheight

% Allow tables
\usepackage{multirow}



%% Packages
%% --------

\usepackage[utf8]{inputenc}
\usepackage[english]{babel}
\usepackage{multicol}
\usepackage{setspace}

\usepackage{amsmath,amssymb,amsfonts}
\usepackage{mathrsfs}

%\usepackage{eurosym}
\usepackage[autolanguage]{numprint}
\usepackage[squaren,Gray]{SIunits}
\usepackage{textcomp}

\usepackage{graphicx} % trim = {left, lower, right, upper}
%\usepackage{wrapfig}
% \usepackage{caption}
% \usepackage{subcaption}
% \usepackage{subfigure} is an old package
% \usepackage{subcaption} is not compatible with all beamer version
%\usepackage{subfig} % is supposed to be the best, but its use is different

\usepackage[skip=0.01pt]{caption} % example skip set to 2pt
%\usepackage{caption}
\captionsetup{labelformat=empty,labelsep=none}
\usepackage[skip=0.01pt]{subcaption}
\usepackage{capt-of}
\captionsetup[subfigure]{labelformat = parens, labelsep = space, labelfont=bf}
\captionsetup[sub]{labelsep=newline}
\setbeamertemplate{caption}[numbered]
     %Change the subcaptions to capital letters
%\renewcommand{\thesubfigure}{\Alph{subfigure}}
%\usepackage{graphicx,subcaption}
\newcommand\newsubcap[1]{\phantomcaption%
       \caption*{\figurename~\thefigure(\thesubfigure): #1}}

\setbeamercovered{dynamic}

\captionsetup[subfigure]{labelformat=empty}




%\usepackage[colorlinks]{hyperref}
\usepackage{hyperref}
\usepackage{animate}[2017/05/18]
%\setbeamercovered{dynamic} % not yet displayed objects are transparent: does not work with images.
\usepackage{xparse} %if in function
%\usepackage{psnfss}
\usepackage[export]{adjustbox}

\usepackage{appendixnumberbeamer}

% Franck packages
% ----------------

%\usepackage{ae, aeguill}
%\usepackage{stmaryrd}
%\usepackage{times} % times font (larger than standard)
%\usepackage{color}
%\usepackage{enumerate}
%\usepackage[all]{xy}    % for commutative diagrams
%\usepackage{pgf, pgfarrows, pgfautomata, pgfheaps, pgfnodes, pgfshade}
%\usepackage{multimedia}   % for embedding movies
%\usepackage{symbmath}
%\usepackage{ifthen}
%\usepackage[FIGTOPCAP]{subfigure}
%\usepackage{pifont}



% Introduction des logos
% ----------------------

\logo{
%	\begin{columns}
%		\hspace{-32pt}
%		\column[b]{0.65\paperwidth}
		%\scalebox{0.8}{\textcolor{my_lightgray}{SVI - CONFIDENTIAL Disclosure or reproduction without prior written permission of Saint-Gobain Research Paris is prohibited}}
%		\column[b]{0.035\paperwidth}
		%\putscale(3){logos/Logos}{0.13}
	%	\vspace{-3pt}
%		\column[b]{0.11\paperwidth}
		%\includegraphics[page=5]{logos/Logos}
	%	\vspace{-8pt}
%		\column[b]{0.1\paperwidth}
		{\scriptsize \hyperlink{tableofcontent}{\textcolor{my_darkgray}{\insertframenumber/\inserttotalframenumber}}}
%	\end{columns}
	\vspace{-5pt}
}

% Background
% ----------

%\setbeamertemplate{background}% {
% \ifthenelse{\value{page}=1}{
% \includegraphics[page=7]{logos/Logos}
% }
% %  \includegraphics{}
% % }
% %else
% %	{}%\hspace{0.5cm}\includegraphics[width=0.06\paperwidth, page=6]{logos/Logos}}
% }

% %\setbeamertemplate{frametitle}{
% %	\vspace{0.2cm}\hspace{0.45cm} \insertframetitle
% %}

% \setbeamertemplate{frametitle}{
%  % \vspace{0.2cm}
%   \insertframetitle
% }

%% New commands
%% ------------

\DeclareMathOperator{\grad}{\nabla}
\DeclareMathOperator{\dive}{\nabla\cdot}
\DeclareMathOperator{\rot}{rot}
\DeclareMathOperator{\laplacien}{\Delta}

%\newcommand{\paragraphtitle}[1]{{\large #1 \smallskip}}
\newcommand{\paragraphtitle}[1]{{\hspace{-0.35cm} \large #1 \smallskip}}
\newcommand{\semitransp}[2][35]{{\color{fg!#1}#2}}
\newcommand{\setframetitle}{\hspace{-0.35cm}\semitransp[50]{\insertsectionhead{}\newline $\blacktriangleright{}$}\insertsubsectionhead{}\vspace{0.2cm}}

\newcommand{\dev}{\text{d}}
\newcommand{\deriv}[2]{\frac{\dev #1}{\dev #2}}
\newcommand{\depar}[2]{\frac{\partial #1}{\partial #2}}
\newcommand{\delag}[2]{\frac{\mathcal{D} #1}{\mathcal{D} #2}}
\renewcommand{\vec}[1]{\mathbf{#1}}
\newcommand{\mat}[1]{\underline{\underline{#1}}}

\newcommand{\ind}[1]{_\text{#1}}
\newcommand{\expo}[1]{^\text{#1}}
\renewcommand{\thefootnote}{\arabic{footnote}}
\newcommand{\val}[2]{\numprint{#1}~\text{#2}}

\newcommand{\para}{\medbreak\par}
\renewcommand{\epsilon}{\varepsilon}
\renewcommand{\phi}{\varphi}
\newcommand{\up}{\textsuperscript}

\newcommand{\aaj}[1]{\textcolor{red}{[#1]}}
\newcommand{\mycite}[1]{
	{\footnotesize \par \textcolor{my_main}{#1} \par}
	\vspace{-2pt}}

\DeclareDocumentCommand \putput { d[] d[] d() m m } {%
	\IfNoValueTF{#3}{%
		\includegraphics[width=#5\textwidth]{#4}%
	}{%
		\includegraphics[page=#3, width=#5\textwidth]{#4}%
	}%
	\IfNoValueTF{#1}{}{\caption{#1}}
	\IfNoValueTF{#2}{}{\label{#2}}
}

\DeclareDocumentCommand \putscale { d<> d<> d() m m } {%
	\IfNoValueTF{#3}{%
		\includegraphics[scale=#5]{#4}%
	}{%
		\includegraphics[page=#3, scale=#5]{#4}%
	}%
	\IfNoValueTF{#1}{}{\caption{#1}}
	\IfNoValueTF{#2}{}{\label{#2}}
}

\DeclareDocumentCommand \video { d[] d[] m m m m m } {%
	% graphics option, video options, speed, start, end, best, name
	\ifthenelse{\final=1}{% video
		\animategraphics[#1, #2]{#3}{#7}{#4}{#5}%
	}{%image
		\includegraphics[#1, clip]{#7#6}%
	}%
}%

\newcommand{\cfbox}[1]{%
	\colorlet{currentcolor}{.}%
	{\color{my_gray}%
		\fbox{#1}}%
}%

\DeclareDocumentCommand \videobox { d[] d[] m m m m m } {%
	% graphics option, video options, speed, start, end, best, name
	\ifthenelse{\final=1}{% video
		\cfbox{\animategraphics[#1, #2]{#3}{#7}{#4}{#5}}%
	}{%image
		\cfbox{\includegraphics[#1, clip]{#7#6}}%
	}%
}%

\newcommand{\ar}{$\blacktriangleright$}

\newcommand{\liquid}[1]{\textcolor{mixture}{#1_{\mathcal{L}}}}
\newcommand{\evapliquid}[1]{\textcolor{water}{#1_{\mathcal{L}_1}}}
\newcommand{\notevapliquid}[1]{\textcolor{propylene}{#1_{\mathcal{L}_2}}}
\newcommand{\vapor}[1]{\textcolor{my_purple}{#1_{\mathcal{V}}}}

\newcommand{\setnormalfigurespace}{
	\setlength{\topsep}{5pt}
	\setlength{\abovecaptionskip}{5pt}
	\setlength{\belowcaptionskip}{0pt}
}

\setnormalfigurespace{}
\setlength{\parindent}{0cm}

\setlength{\fboxsep}{0.1pt}%
\setlength{\fboxrule}{0.2pt}%


%\usepackage{enumitem}
%\setlist[itemize]{leftmargin=*}





%Use tikz to scale credit label
%\usepackage{tikz}
\usetikzlibrary{mindmap,trees,shadows}
\usetikzlibrary{calc}
\tikzset{credit/.style={
    append after command={
      let \p1=($(\tikzlastnode.north east)-(\tikzlastnode.north west)$) in
      node[anchor=north east,align=right,text width=\x1,font=\footnotesize,
        node contents={\scalebox{0.45}{Credit:~\textit{#1}}}, at = (\tikzlastnode.south east)]

    }
  }
}
\def\myimage#1#2#3{\tikz\node[inner sep=0,credit={#3}]{\includegraphics[#1]{#2}};}

\def\myimageCopy#1#2#3{\tikz\node[inner sep=0,credit={#3}]{\includegraphics[#1]{#2}};}




% To make a colored box around the text:
\usepackage[framemethod=TikZ]{mdframed}
\usepackage{lipsum}
\mdfdefinestyle{MyFrame}{%
    linecolor=my_raspberry,
    outerlinewidth=2pt,
    roundcorner=1pt,
    innertopmargin=\baselineskip,
    innerbottommargin=\baselineskip,
    innerrightmargin=2pt,
    innerleftmargin=2pt,
    backgroundcolor=white}


%Set footnote without marker for the last slide:
%To reset the footnote counter in the last slide
%\usepackage{perpage}
\usepackage{appendixnumberbeamer}


% appendixframe includes a hyperlink to the frame labeled ‘questions’.
% The ‘\vskip’ commands are only included so that this button always
% appears on the bottom of the frame.
\newenvironment{appendixframe}[1]
  {\begin{frame}[environment=appendixframe]\frametitle{#1}}
  {\vskip0ptplus1filll%
    \hyperlink{questions}{\beamerreturnbutton{Return}}%
    \vskip2ex\end{frame}}



% %Redefine the footnote command to have a non-numbering option
% \let\svthefootnote\thefootnote
% \textheight 1in
% \newcommand\blankfootnote[1]{%
%   \let\thefootnote\relax\footnotetext{#1}%
%   \let\thefootnote\svthefootnote%
% }
% \let\svfootnote\footnote
% \renewcommand\footnote[2][?]{%
%   \if\relax#1\relax%
%     \blankfootnote{#2}%
%   \else%
%     \if?#1\svfootnote{#2}\else\svfootnote[#1]{#2}\fi%
%   \fi
% }


%Add text in bottom of slide:

\newcommand{\btVFill}{\vskip0pt plus 1filll}


%Italic font style
\newcommand*\latin[1]{\textit{#1}}


% Usepackage to cancel mathematical expressions
\usepackage{cancel}

%To color cancel in red:
\newcommand\Ccancel[2][red]{\renewcommand\CancelColor{\color{#1}}\xcancel{#2}}



% --- Bold math symbol
\usepackage{bm}
